% Options for packages loaded elsewhere
\PassOptionsToPackage{unicode}{hyperref}
\PassOptionsToPackage{hyphens}{url}
\PassOptionsToPackage{dvipsnames,svgnames,x11names}{xcolor}
%
\documentclass[
  a4paper,
]{report}

\usepackage{amsmath,amssymb}
\usepackage{iftex}
\ifPDFTeX
  \usepackage[T1]{fontenc}
  \usepackage[utf8]{inputenc}
  \usepackage{textcomp} % provide euro and other symbols
\else % if luatex or xetex
  \usepackage{unicode-math}
  \defaultfontfeatures{Scale=MatchLowercase}
  \defaultfontfeatures[\rmfamily]{Ligatures=TeX,Scale=1}
\fi
\usepackage{lmodern}
\ifPDFTeX\else  
    % xetex/luatex font selection
\fi
% Use upquote if available, for straight quotes in verbatim environments
\IfFileExists{upquote.sty}{\usepackage{upquote}}{}
\IfFileExists{microtype.sty}{% use microtype if available
  \usepackage[]{microtype}
  \UseMicrotypeSet[protrusion]{basicmath} % disable protrusion for tt fonts
}{}
\makeatletter
\@ifundefined{KOMAClassName}{% if non-KOMA class
  \IfFileExists{parskip.sty}{%
    \usepackage{parskip}
  }{% else
    \setlength{\parindent}{0pt}
    \setlength{\parskip}{6pt plus 2pt minus 1pt}}
}{% if KOMA class
  \KOMAoptions{parskip=half}}
\makeatother
\usepackage{xcolor}
\setlength{\emergencystretch}{3em} % prevent overfull lines
\setcounter{secnumdepth}{5}
% Make \paragraph and \subparagraph free-standing
\makeatletter
\ifx\paragraph\undefined\else
  \let\oldparagraph\paragraph
  \renewcommand{\paragraph}{
    \@ifstar
      \xxxParagraphStar
      \xxxParagraphNoStar
  }
  \newcommand{\xxxParagraphStar}[1]{\oldparagraph*{#1}\mbox{}}
  \newcommand{\xxxParagraphNoStar}[1]{\oldparagraph{#1}\mbox{}}
\fi
\ifx\subparagraph\undefined\else
  \let\oldsubparagraph\subparagraph
  \renewcommand{\subparagraph}{
    \@ifstar
      \xxxSubParagraphStar
      \xxxSubParagraphNoStar
  }
  \newcommand{\xxxSubParagraphStar}[1]{\oldsubparagraph*{#1}\mbox{}}
  \newcommand{\xxxSubParagraphNoStar}[1]{\oldsubparagraph{#1}\mbox{}}
\fi
\makeatother


\providecommand{\tightlist}{%
  \setlength{\itemsep}{0pt}\setlength{\parskip}{0pt}}\usepackage{longtable,booktabs,array}
\usepackage{calc} % for calculating minipage widths
% Correct order of tables after \paragraph or \subparagraph
\usepackage{etoolbox}
\makeatletter
\patchcmd\longtable{\par}{\if@noskipsec\mbox{}\fi\par}{}{}
\makeatother
% Allow footnotes in longtable head/foot
\IfFileExists{footnotehyper.sty}{\usepackage{footnotehyper}}{\usepackage{footnote}}
\makesavenoteenv{longtable}
\usepackage{graphicx}
\makeatletter
\def\maxwidth{\ifdim\Gin@nat@width>\linewidth\linewidth\else\Gin@nat@width\fi}
\def\maxheight{\ifdim\Gin@nat@height>\textheight\textheight\else\Gin@nat@height\fi}
\makeatother
% Scale images if necessary, so that they will not overflow the page
% margins by default, and it is still possible to overwrite the defaults
% using explicit options in \includegraphics[width, height, ...]{}
\setkeys{Gin}{width=\maxwidth,height=\maxheight,keepaspectratio}
% Set default figure placement to htbp
\makeatletter
\def\fps@figure{htbp}
\makeatother

\makeatletter
\@ifpackageloaded{tcolorbox}{}{\usepackage[skins,breakable]{tcolorbox}}
\@ifpackageloaded{fontawesome5}{}{\usepackage{fontawesome5}}
\definecolor{quarto-callout-color}{HTML}{909090}
\definecolor{quarto-callout-note-color}{HTML}{0758E5}
\definecolor{quarto-callout-important-color}{HTML}{CC1914}
\definecolor{quarto-callout-warning-color}{HTML}{EB9113}
\definecolor{quarto-callout-tip-color}{HTML}{00A047}
\definecolor{quarto-callout-caution-color}{HTML}{FC5300}
\definecolor{quarto-callout-color-frame}{HTML}{acacac}
\definecolor{quarto-callout-note-color-frame}{HTML}{4582ec}
\definecolor{quarto-callout-important-color-frame}{HTML}{d9534f}
\definecolor{quarto-callout-warning-color-frame}{HTML}{f0ad4e}
\definecolor{quarto-callout-tip-color-frame}{HTML}{02b875}
\definecolor{quarto-callout-caution-color-frame}{HTML}{fd7e14}
\makeatother
\makeatletter
\@ifpackageloaded{bookmark}{}{\usepackage{bookmark}}
\makeatother
\makeatletter
\@ifpackageloaded{caption}{}{\usepackage{caption}}
\AtBeginDocument{%
\ifdefined\contentsname
  \renewcommand*\contentsname{Table of contents}
\else
  \newcommand\contentsname{Table of contents}
\fi
\ifdefined\listfigurename
  \renewcommand*\listfigurename{List of Figures}
\else
  \newcommand\listfigurename{List of Figures}
\fi
\ifdefined\listtablename
  \renewcommand*\listtablename{List of Tables}
\else
  \newcommand\listtablename{List of Tables}
\fi
\ifdefined\figurename
  \renewcommand*\figurename{Figure}
\else
  \newcommand\figurename{Figure}
\fi
\ifdefined\tablename
  \renewcommand*\tablename{Table}
\else
  \newcommand\tablename{Table}
\fi
}
\@ifpackageloaded{float}{}{\usepackage{float}}
\floatstyle{ruled}
\@ifundefined{c@chapter}{\newfloat{codelisting}{h}{lop}}{\newfloat{codelisting}{h}{lop}[chapter]}
\floatname{codelisting}{Listing}
\newcommand*\listoflistings{\listof{codelisting}{List of Listings}}
\makeatother
\makeatletter
\makeatother
\makeatletter
\@ifpackageloaded{caption}{}{\usepackage{caption}}
\@ifpackageloaded{subcaption}{}{\usepackage{subcaption}}
\makeatother

\ifLuaTeX
  \usepackage{selnolig}  % disable illegal ligatures
\fi
\usepackage{bookmark}

\IfFileExists{xurl.sty}{\usepackage{xurl}}{} % add URL line breaks if available
\urlstyle{same} % disable monospaced font for URLs
\hypersetup{
  pdftitle={A-Level Further Maths (Edexcel) - Unit tests 2024-25 model answers},
  pdfauthor={Evi Papadaki},
  colorlinks=true,
  linkcolor={blue},
  filecolor={Maroon},
  citecolor={Blue},
  urlcolor={Blue},
  pdfcreator={LaTeX via pandoc}}


\title{A-Level Further Maths (Edexcel) - Unit tests 2024-25 model
answers}
\author{Evi Papadaki}
\date{2024-10-02}

\begin{document}
\maketitle

\renewcommand*\contentsname{Table of contents}
{
\hypersetup{linkcolor=}
\setcounter{tocdepth}{2}
\tableofcontents
}

\bookmarksetup{startatroot}

\chapter*{Index}\label{index}
\addcontentsline{toc}{chapter}{Index}

\markboth{Index}{Index}

The booklet is organised by topic. Use the list on the left of your
screen to navigate the different chapters.

\begin{tcolorbox}[enhanced jigsaw, title=\textcolor{quarto-callout-tip-color}{\faLightbulb}\hspace{0.5em}{Tip}, titlerule=0mm, colbacktitle=quarto-callout-tip-color!10!white, coltitle=black, toprule=.15mm, colframe=quarto-callout-tip-color-frame, breakable, bottomtitle=1mm, toptitle=1mm, arc=.35mm, rightrule=.15mm, bottomrule=.15mm, left=2mm, opacitybacktitle=0.6, leftrule=.75mm, opacityback=0, colback=white]

You can use the \emph{search tool} to look for a specific concept.

\end{tcolorbox}

\bookmarksetup{startatroot}

\chapter{Series - CP Chapter 2}\label{series---cp-chapter-2}

\section{Question 1}\label{question-1}

\begin{enumerate}
\def\labelenumi{\alph{enumi}.}
\item
  \textbf{(5 marks)} Prove that
  \[\sum_{r = 1}^{n}{\frac{3}{r(r + 1)} = \frac{an}{n + 1},\ n} \in \mathbb{Z}\]
  where \(a\) is a constant to be found.
\item
  \textbf{(1 mark)} Find the value of
  \(\sum_{r = 1}^{50}{\frac{3}{r(r + 1)}}\), giving your answer as an
  exact fraction.
\item
  \textbf{(4 marks)} Find an expression in its simplest form for
\end{enumerate}

\[\sum_{r = n}^{2n}{\frac{3}{r(r + 1)}}\]\\

\begin{tcolorbox}[enhanced jigsaw, title={Solution}, titlerule=0mm, colbacktitle=quarto-callout-tip-color!10!white, coltitle=black, toprule=.15mm, colframe=quarto-callout-tip-color-frame, breakable, bottomtitle=1mm, toptitle=1mm, arc=.35mm, rightrule=.15mm, bottomrule=.15mm, left=2mm, opacitybacktitle=0.6, leftrule=.75mm, opacityback=0, colback=white]

\textbf{a} We notice that
\begin{equation}\phantomsection\label{eq-1a}{\sum_{r=1}^{n}\frac{3}{r(r+1)}=3\sum_{r=1}^{n}\frac{1}{r(r+1)}}\end{equation}
First, we need to express \(\frac{1}{r(r+1)}\) in partial fractions. \[
\frac{1}{r(r+1)}=\frac{A}{r}+\frac{B}{r+1}\] for some
\(A,B \in \mathbb{R}\).

By multiplying both sides by the common denominator \(r(r+1)\), we have:

\[1=A(r+1)+Br \] For \(r=0\), we get \(A=1\) and for \(r=-1\), \(B=-1\).

\[\therefore \qquad \frac{1}{r(r+1)}=\frac{1}{r}-\frac{1}{r+1} \] So,
from Equation~\ref{eq-1a} we have

\(\sum_{r=1}^{n}\frac{3}{r(r+1)} \ \quad  = \quad 3\sum_{r=1}^{n} \left(\frac{1}{r}-\frac{1}{r+1}\right)\)

\begin{alignat*}{3}
= \quad & 3 & \textcolor{magenta}{-} \qquad & \qquad  \textcolor{magenta}{\frac{3}{2}}  & \quad (r=1) \\
\textcolor{magenta}{+}  & \textcolor{magenta}{\frac{3}{2}}  &\textcolor{magenta}{-} \qquad& \qquad \textcolor{magenta}{ 1} & \quad (r=2) \\
\textcolor{magenta}{+} &\textcolor{magenta}{1} &\textcolor{magenta}{-} \qquad& \qquad  \textcolor{magenta}{\frac{3}{4}} & \quad (r=3) \\
&\textcolor{magenta}{\vdots} & & \quad & \\
\textcolor{magenta}{+}  & \textcolor{magenta}{\frac{3}{n-1}}  &\textcolor{magenta}{-} \qquad& \qquad  \textcolor{magenta}{\frac{3}{n}}  & \quad (r=n-1)\\
\textcolor{magenta}{+} & \textcolor{magenta}{\frac{3}{n}} & - \qquad & \qquad \frac{3}{n+1} & \quad (r=n)\\
= \quad & 3 - \frac{3}{n+1} & & \quad & 
\end{alignat*}

Thus, by simplifying
\begin{equation}\phantomsection\label{eq-1b}{\sum_{r=1}^{n}\frac{3}{r(r+1)}=\frac{3n}{n+1}, \quad \forall n \in \mathbb{Z} }\end{equation}
Therefore, \(a=3\).

\textbf{b} By substituting \(n=50\) in Equation~\ref{eq-1b}
\[\sum_{r=1}^{50}\frac{3}{r(r+1)}=\frac{3\times50}{50+1}=\frac{3 \times 50}{51}=\frac{3\times50}{3 \times17}=\frac{50}{17}\]

\textbf{c} We need to make the following observation about the range of
values for r:
\[\sum_{r=n}^{2n}\frac{3}{r(r+1)}=\sum_{r=1}^{2n}\frac{3}{r(r+1)}-\sum_{r=1}^{n-1}\frac{3}{r(r+1)}\]
By substituting the appropriate values in Equation~\ref{eq-1b}
\begin{equation*}
\begin{split}
\sum_{r=n}^{2n}\frac{3}{r(r+1)} &= \frac{3 \times 2n}{2n+1}-\frac{3(n-1)}{n-1+1} \\
&= \frac{6n}{2n+1}-\frac{3n-3}{n} \\
&= \frac{6n^2-(3n-3)(2n+1)}{n(2n+1)}\\
&= \frac{6n^2-(6n^2-3n-3)}{n(2n+1)}\\
&=\frac{3n+3}{n(2n+1)}\\
&=\frac{3(n+1)}{n(2n+1)}
\end{split}
\end{equation*}

\end{tcolorbox}

\section{Question 2}\label{question-2}

\begin{enumerate}
\def\labelenumi{\alph{enumi}.}
\tightlist
\item
  \textbf{(2 marks)} Simplify
\end{enumerate}

\[r^2\ (r+1)^2-(r-1)^2\ r^2\]

\begin{enumerate}
\def\labelenumi{\alph{enumi}.}
\setcounter{enumi}{1}
\tightlist
\item
  \textbf{(3 marks)} Use the method of differences to show that
  \(\sum_{r=1}^{n}r^3=\frac{1}{4}n^2(n+1)^2\)
\end{enumerate}

\begin{tcolorbox}[enhanced jigsaw, title={Solution}, titlerule=0mm, colbacktitle=quarto-callout-tip-color!10!white, coltitle=black, toprule=.15mm, colframe=quarto-callout-tip-color-frame, breakable, bottomtitle=1mm, toptitle=1mm, arc=.35mm, rightrule=.15mm, bottomrule=.15mm, left=2mm, opacitybacktitle=0.6, leftrule=.75mm, opacityback=0, colback=white]

\textbf{a} \begin{equation*}
\begin{split}
r^2 (r+1)^2-(r-1)^2 r^2 &= r^2 (r^2+2r+1)-(r^2-2r+1)r^2 \\
&= r^4+2r^3+r^2-(r^4-2r^3+r^2) \\
&= r^4+2r^3+r^2-r^4+2r^3-r^2 \\
&= 4r^3 \\
\end{split}
\end{equation*}

\textbf{b} From part a. we can deduce that
\[r^3=\frac{1}{4}(r^2 (r+1)^2-(r-1)^2 r^2)\] Therefore,

\(\sum_{r=1}^{n}r^3  \quad = \quad \frac{1}{4}\sum_{r=1}^{n} \left(r^2 (r+1)^2-(r-1)^2 r^2\right)\)
\begin{alignat*}{3}
= \quad & \textcolor{magenta}{1} &  -  & \qquad  0  & \quad (r=1) \\
\textcolor{magenta}{+}  & \textcolor{magenta}{\frac{2^2 \times 3^2}{4}}  &\textcolor{magenta}{-} &\qquad \textcolor{magenta}{1} & \quad (r=2) \\
\textcolor{magenta}{+} &\textcolor{magenta}{\frac{3^2 \times 4^2}{4}} &\textcolor{magenta}{-} &\qquad  \textcolor{magenta}{\frac{2^2 \times 3^2}{4}} & \quad (r=3) \\
&\textcolor{magenta}{\vdots} & & \quad & \\
\textcolor{magenta}{+}  & \textcolor{magenta}{\frac{(n-1)^2n^2}{4}}  &\textcolor{magenta}{-} & \qquad  \textcolor{magenta}{\frac{(n-2)^2(n-1)^2}{4}}  & \quad (r=n-1)\\
+ & \frac{n^2(n+1)^2}{4} \quad & -  & \qquad \textcolor{magenta}{\frac{(n-1)^2n^2}{4}} & \quad (r=n)\\
= \quad & \frac{n^2(n+1)^2}{4} & & \quad & 
\end{alignat*} Thus, \[\sum_{r=1}^{n}r^3=\frac{1}{4}n^2(n+1)^2\]

\end{tcolorbox}

\section{Question 3}\label{question-3}

\begin{enumerate}
\def\labelenumi{\alph{enumi}.}
\tightlist
\item
  \textbf{(3 marks)} Express in partial fractions
\end{enumerate}

\[\frac{2}{(r+2)(r+3)(r+4)}\]

\begin{enumerate}
\def\labelenumi{\alph{enumi}.}
\setcounter{enumi}{1}
\tightlist
\item
  \textbf{(5 marks)} Show that
  \[\sum_{r=1}^{n}{\frac{2}{(r+2)(r+3)(r+4)}=\frac{n(n+b)}{c(n+3)(n+4)}}\]
  where \(b\) and \(c\) are constants to be found.
\end{enumerate}

\begin{tcolorbox}[enhanced jigsaw, title={Solution}, titlerule=0mm, colbacktitle=quarto-callout-tip-color!10!white, coltitle=black, toprule=.15mm, colframe=quarto-callout-tip-color-frame, breakable, bottomtitle=1mm, toptitle=1mm, arc=.35mm, rightrule=.15mm, bottomrule=.15mm, left=2mm, opacitybacktitle=0.6, leftrule=.75mm, opacityback=0, colback=white]

\textbf{a}
\[\frac{2}{(r+2)(r+3)(r+4)} = \frac{A}{r+2}+\frac{B}{r+3}+\frac{C}{r+4}\]
for some \(A, B, C \in \mathbb{R}\). By multiplying both sides by the
common denominator \((r+2)(r+3)(r+4)\), we have
\[2=A(r+3)(r+4)+B(r+2)(r+4)+C(r+2)(r+3)\] For \(r=-2\), we get \(A=1\),
for \(r=-3\), \(B=-2\) and for \(r=-4\), \(C=1\).

\begin{equation}\phantomsection\label{eq-2a}{\therefore \qquad \frac{2}{(r+2)(r+3)(r+4)} = \frac{1}{r+2}-\frac{2}{r+3}+\frac{1}{r+4} }\end{equation}

\textbf{b} Using the result Equation~\ref{eq-2a} from part a

\(\sum_{r=1}^{n}\frac{2}{(r+2)(r+3)(r+4)} = \sum_{r=1}^{n}\frac{1}{r+2}-\frac{2}{r+3}+\frac{1}{r+4}\)
\begin{alignat*}{4}
= \quad & \qquad & \frac{1}{3}&  \qquad - &\qquad \frac{2}{4} & \qquad + & \qquad   \textcolor{magenta}{\frac{1}{5}}  & \quad (r=1) \\
& + & \quad  \frac{1}{4}& \qquad - & \qquad \textcolor{magenta}{\frac{2}{5}} & \qquad + & \qquad  \textcolor{blue}{\frac{1}{6}}  & \quad (r=2)\\
& + & \quad \textcolor{magenta}{\frac{1}{5}} & \qquad - & \qquad \textcolor{blue}{\frac{2}{6}} & \qquad + & \qquad  \textcolor{magenta}{\frac{1}{7}}  & \quad (r=3)\\
& + & \quad \textcolor{blue}{\frac{1}{6}} & \qquad - & \qquad \textcolor{magenta}{\frac{2}{7}} & \qquad + & \qquad  \textcolor{blue}{\frac{1}{8}}  & \quad (r=4)\\
&\vdots & & \quad & \\
& + & \quad \textcolor{magenta}{\frac{1}{n}}& \qquad - & \textcolor{blue}{\qquad \frac{2}{n+1}} & \qquad + & \qquad \textcolor{magenta}{\frac{1}{n+2}}  & \quad (r=n-2)\\
& + & \quad  \textcolor{blue}{\frac{1}{n+1}}& \qquad - & \textcolor{magenta}{\qquad \frac{2}{n+2}} & \qquad + & \qquad  \frac{1}{n+3}  & \quad (r=n-1)\\
& + & \quad  \textcolor{magenta}{\frac{1}{n+2}}& \qquad - & \qquad \frac{2}{n+3} & \qquad + & \qquad  \frac{1}{n+5}  & \quad (r=n) 
\end{alignat*}

Thus,

\begin{equation*}
\begin{split}
\sum_{r=1}^{n}\frac{2}{(r+2)(r+3)(r+4)}&= \frac{1}{2}-\frac{2}{4}+\frac{1}{4}+\frac{1}{n+3}-\frac{2}{n+4}+\frac{1}{n+4}\\
&=\frac{1}{12}-\frac{1}{n+3}+\frac{1}{n+4}\\
&= \frac{(n+3)(n+4)-12(n+4)+12(n+3)}{12(n+3)(n+4)}\\
&= \frac{n^2+7n+12-12n-48+12n+36}{12(n+3)(n+4)}\\
&= \frac{n^2+7n}{12(n+3)(n+4)}\\
&= \frac{n(n+7)}{12(n+3)(n+4)}
\end{split}
\end{equation*}

Thus, \(b=7\) and \(c=12\).

\end{tcolorbox}

\section{Question 4}\label{question-4}

\begin{enumerate}
\def\labelenumi{\alph{enumi}.}
\tightlist
\item
  \textbf{(3 marks)} Use standard results to show that the first four
  terms of the series expansion of \(e^{2-x}\) in ascending powers of
  \(x\) can be expressed as
\end{enumerate}

\[e^2\left(1-x+\frac{x^2}{2!}-\frac{x^3}{3!}\right)\]

\begin{enumerate}
\def\labelenumi{\alph{enumi}.}
\setcounter{enumi}{1}
\tightlist
\item
  \textbf{(3 marks)} Use standard results to obtain the first four
  non-zero terms of the series expansion of
\end{enumerate}

\[sin{(3x^2)}\]

\begin{enumerate}
\def\labelenumi{\alph{enumi}.}
\setcounter{enumi}{2}
\tightlist
\item
  \textbf{(4 marks)} Use standard results to show that for all real
  values of \(x\)
\end{enumerate}

\[\cosh^{2}{x \geq 1 + x^{2}}\]

\begin{tcolorbox}[enhanced jigsaw, title={Solution}, titlerule=0mm, colbacktitle=quarto-callout-tip-color!10!white, coltitle=black, toprule=.15mm, colframe=quarto-callout-tip-color-frame, breakable, bottomtitle=1mm, toptitle=1mm, arc=.35mm, rightrule=.15mm, bottomrule=.15mm, left=2mm, opacitybacktitle=0.6, leftrule=.75mm, opacityback=0, colback=white]

\textbf{a} First we notice that the requested expression has a factor of
\(e^2\). Because Maclaurin series approximate \(e^t\) as a polynomial
series, it means that we need to factorise before applying the standard
rule for \(e^t\).

Using laws of indices, we get
\begin{equation}\phantomsection\label{eq-4a}{e^{2-x}=e^2e^{-x}}\end{equation}
Now we apply the standard result for
\(e^t=\sum_{r=0}^\infty\frac{t^r}{r!}\), \(\forall t\in \mathbb{R}\) for
\(t=-x\) to obtain the first four terms of the series expansion for
\(e^{-x}\). \begin{equation*}
\begin{split}
e^{-x}&=1+(-x)+\frac{(-x)^2}{2!}+\frac{(-x)^3}{3!} \\
&=1-x+\frac{x^2}{2!}-\frac{x^3}{3!}
\end{split}
\end{equation*} Thus, from Equation~\ref{eq-1a}
\[e^{2-x}=e^2\left(1-x+\frac{x^2}{2!}-\frac{x^3}{3!}\right)\] \textbf{b}
We use the standard results
\(\sin t = \sum_{r=0}^\infty (-1)^r \frac{t^{2r+1}}{(2r+1)!}\),
\(\forall t \in \mathbb{R}\) for \(t=3x^2\), to find the first four
non-zero terms.

\begin{equation*}
\begin{split}
\sin 3x^2 &= 3x^2 - \frac{(3x^2)^3}{3!}+ \frac{(3x^2)^5}{5!}-\frac{(3x^2)^7}{7!} \\
&=3x^2 -\frac{3^3}{3!}x^6+\frac{3^5}{5!}x^{10}-\frac{3^7}{7!}x^{14} \\
&=3x^2 -\frac{9}{2}x^6+\frac{81}{40}x^{10}-\frac{243}{560}x^{14}
\end{split}
\end{equation*}

\textbf{c}\\
\begin{equation}\phantomsection\label{eq-4b}{\cosh x = \frac{1}{2}(e^x+e^{-x})}\end{equation}
We need to consider the standard result for \(e^t\).

\begin{equation*}
\begin{split}
e^{x}&=1+2x+\frac{x^2}{2!}+\frac{x^3}{3!}+\frac{x^4}{4!} \quad ... \\
e^{-x}&=1+(-x)+\frac{(-x)^2}{2!}+\frac{(-x)^3}{3!}+\frac{(x)^4}{4!} \quad ... \\
&=1-x+\frac{(x)^2}{2!}-\frac{(x)^3}{3!}+\frac{(x)^4}{4!} \quad ... \\
\therefore \qquad e^{x}+e^{-x}&=2+2\frac{(x)^2}{2!}+2\frac{(x)^4}{4!}+ \quad ...
\end{split}
\end{equation*}

By substituting this result in Equation~\ref{eq-4b}, we get

\begin{equation*}
\begin{split}
\cosh x &= \frac{1}{2} \left( 2+2\frac{(x)^2}{2!}+2\frac{(x)^4}{4!}+ \quad ... \right)\\
&= 1+\frac{(x)^2}{2!}+\textcolor{magenta}{\frac{(x)^4}{4!}+ \quad ...}
\end{split}
\end{equation*} The residual(you don't need to know how it's called)
\[\frac{(x)^4}{4!}+ \quad ... \geq 0, \quad \forall x \in \mathbb{R}\]
as all terms are of even degree with positive coefficients.

Therefore, \[\cosh x \geq1+\frac{x^2}{2}\] So, \begin{equation*}
\begin{split}
\cosh^2 x &\geq \left(1+\frac{x^2}{2}\right)^2 \\
\cosh^2 x &\geq 1+2\frac{x^2}{2}+\frac{x^4}{4} \\
 &\geq 1+x^2
\end{split}
\end{equation*} as \(\frac{x^4}{4}\geq0\), \(\forall x \in \mathbb{R}\).

\textbf{Alternative method:} This method is slightly longer but it
requires less planning.

First, \begin{equation*}
\begin{split}
\cosh^2 x &= \left( \frac{1}{2}(e^x+e^{-x}) \right) ^2 \\
&= \frac{1}{4}(e^x+e^{-x})^2 \\
&= \frac{1}{4}(e^{2x}+2e^x e^{-x}+e^{-2x}) \\
&= \frac{1}{4}(e^{2x}+2+e^{-2x})
\end{split}
\end{equation*}

Thus,
\begin{equation}\phantomsection\label{eq-4c}{\cosh^2x=\frac{1}{2}+\frac{1}{4}(e^{2x}+e^{-2x})}\end{equation}

Then we need to consider the standard result for \(e^t\).

{[}\textbf{Note:} to prove the inequality \(\forall x \in \mathbb{R}\)
using this method, we need to consider the Maclaurin series of
\(\cosh^2 x\). An approximation is not enough, because we do not know if
the approximation underestimates or overestimates the true values.{]}

\begin{equation*}
\begin{split}
e^{2x}&=1+2x+\frac{(2x)^2}{2!}+\frac{(2x)^3}{3!}+\frac{(2x)^4}{4!} \quad ... \\
e^{-2x}&=1+(-2x)+\frac{(-2x)^2}{2!}+\frac{(-2x)^3}{3!}+\frac{(2x)^4}{4!} \quad ... \\
&=1-2x+\frac{(2x)^2}{2!}-\frac{(2x)^3}{3!}+\frac{(2x)^4}{4!} \quad ... \\
\therefore \qquad e^{2x}+e^{-2x}&=2+2\frac{(2x)^2}{2!}+2\frac{(2x)^4}{4!}+ \quad ...
\end{split}
\end{equation*} By substituting this result in Equation~\ref{eq-4c}, we
get \begin{equation*}
\begin{split}
\cosh ^2 x &=\frac{1}{2}+\frac{1}{4} \left(2+2\frac{(2x)^2}{2!}+2\frac{(2x)^4}{4!}+ \quad ... \right)\\
&=\frac{1}{2}+\frac{1}{2}+\frac{1}{2}\frac{(2x)^2}{2!}+\frac{1}{2}\frac{(2x)^4}{4!}+ \quad ... \\
&=1+x^2+\textcolor{magenta}{\frac{2}{3}x^4+ \quad ...}
\end{split}
\end{equation*}

The residual(you don't need to know how it's called)
\[\frac{2}{3}x^4+ \quad ... \geq 0, \quad \forall x \in \mathbb{R}\] as
all terms are of even degree with positive coefficients.

Therefore, \[\cosh ^2 x \geq 1+x^2\]

\end{tcolorbox}

\section{Question 5}\label{question-5}

\begin{enumerate}
\def\labelenumi{\alph{enumi}.}
\item
  \textbf{(5 marks}) Show that the series expansion of
  \(ln{\left(\frac{1+3x}{1-2x}\right)}\) in ascending powers of \(x\),
  up to and including the term in \(x^4\), is
  \[\frac{5x^2}{2}+\frac{35x^3}{3}-\frac{65x^4}{4}\]
\item
  \textbf{(1 mark)} State the range of values of \(x\) for which the
  answer to part \textbf{a} is valid.
\item
  \textbf{(4 marks)} By choosing a suitable value for \(x\), use the
  expansion from part \textbf{a} to obtain an estimate for the value of
  \(ln{\frac{1}{2}}\). Give your answer to 3 decimal places.
\item
  \textbf{(2 marks)} Write down the first four terms of the series
  expansion for \(ln{\sqrt{\frac{1+3x}{1-2x}}}\)
\end{enumerate}

\begin{tcolorbox}[enhanced jigsaw, title={Solution}, titlerule=0mm, colbacktitle=quarto-callout-tip-color!10!white, coltitle=black, toprule=.15mm, colframe=quarto-callout-tip-color-frame, breakable, bottomtitle=1mm, toptitle=1mm, arc=.35mm, rightrule=.15mm, bottomrule=.15mm, left=2mm, opacitybacktitle=0.6, leftrule=.75mm, opacityback=0, colback=white]

\textbf{a} From the laws of logarithms
\begin{equation}\phantomsection\label{eq-5a}{\ln \left( \frac{1+3x}{1-2x}\right) = \ln (1+3x) + \ln (1 -2x)}\end{equation}

From the standard result
\(\ln(1+t) \sum_{r=0}^{\infty}(-1)^{r+1}\frac{t^r}{r}\), for
\(-1 < t \leq 1\), for \(t=3x\) and \(t=-2x\) we've got

\begin{equation*}
\begin{split}
\ln (1+3x) &=3x-\frac{(3x)^2}{2}+\frac{(3x)^3}{3}-\frac{(3x)^4}{4} \\
&= 3x-\frac{9}{2}x^2+9x^3-\frac{81}{4}x^4 \\
\ln (1-2x) &=-2x-\frac{(-2x)^2}{2}+\frac{(-2x)^3}{3}-\frac{(-2x)^4}{4} \\
&=-2x-\frac{(2x)^2}{2}-\frac{(2x)^3}{3}-\frac{(2x)^4}{4} \\
&=-2x-2x^2 -\frac{8}{3}x^3-\frac{16}{4}x^4
\end{split}
\end{equation*}

up to and including \(x^4\).

Thus, from Equation~\ref{eq-5a}
\begin{equation}\phantomsection\label{eq-5b}{\ln \left( \frac{1+3x}{1-2x}\right) \approx 5x-\frac{5}{2}x^2+\frac{35}{3}x^3-\frac{65}{4}x^4}\end{equation}

\textbf{b} We require the approximation to be valid for both the
expansion of \(\ln(1+3x)\) and \(\ln(1-2x)\). Therefore,
Equation~\ref{eq-5b} is valid for \[-\frac{1}{3} < x \leq -\frac{1}{3}\]

\textbf{c} First we need to find the value of x that satisfies the
equality \begin{equation*}
\begin{split}
\frac{1+3x}{1-2x}&=\frac{1}{2} \\
2(1+3x) &=1-2x \\
2+6x&=1-2x\\
8x&=-1\\
x&=-\frac{1}{8}
\end{split}
\end{equation*}

By substituting \(x=-\frac{1}{8}\) in Equation~\ref{eq-5b}

\begin{equation*}
\begin{split}
\ln \left( \frac{1}{2} \right)& \approx  5\left(- \frac{1}{8} \right)-\frac{5}{2}\left( -\frac{1}{8} \right)^2+\frac{35}{3}\left(- \frac{1}{8} \right)^3-\frac{65}{4}\left( -\frac{1}{8} \right)^4 \\
&=-0.6908162435 \\
&= 0.691 \quad \textsf{(3 d.p.)}
\end{split}
\end{equation*}

\textbf{d} By the laws of logarithms

\begin{equation*}
\begin{split}
\ln \left( \sqrt{\frac{1+3x}{1-2x}}\right)&= \left(\ln \left( \frac{1+3x}{1-2x}\right)\right)^{\frac{1}{2}} \\
&=\frac{1}{2}\ln \left( \sqrt{\frac{1+3x}{1-2x}}\right) \\
& \approx \frac{5}{2}x-\frac{5}{4}x^2+\frac{35}{6}x^3-\frac{65}{8}x^4
\end{split}
\end{equation*}

\end{tcolorbox}

\section{Question 6}\label{question-6}

\emph{(from Q2 Paper 1 November 2021)}

\begin{enumerate}
\def\labelenumi{\alph{enumi}.}
\item
  \textbf{(2 marks)} Use the Maclaurin series expansion for \(\cos x\)
  to determine the series expansion of \(\cos^{2}\frac{x}{3}\) in
  ascending powers of \(x\),up to and including the term \(x^{4}\).

  Give each term in simplest form.
\item
  \textbf{(3 marks)} Use the answer to part \emph{a} and calculus to
  find an approximation, to 5 decimal places, for
\end{enumerate}

\[\int_{\pi/6}^{\pi/2}{\left( \frac{1}{x}\cos^{2}\left( \frac{x}{3} \right) \right)dx}\]
Given that \(\int_{}^{}{\frac{1}{x}dx = \ln{x + c}}\).

\begin{tcolorbox}[enhanced jigsaw, title={Solution}, titlerule=0mm, colbacktitle=quarto-callout-tip-color!10!white, coltitle=black, toprule=.15mm, colframe=quarto-callout-tip-color-frame, breakable, bottomtitle=1mm, toptitle=1mm, arc=.35mm, rightrule=.15mm, bottomrule=.15mm, left=2mm, opacitybacktitle=0.6, leftrule=.75mm, opacityback=0, colback=white]

\textbf{a} To simplify the calculations, we can consider the double
angle formula

\[\cos^2 t = \frac{1}{2}(1+\cos 2t)\] Using the standard results for the
expansion of \(\cos t\) \begin{equation*}
\begin{split}
\cos^2 t &= \frac{1}{2}(1+1-\frac{(2t)^2}{2!}+\frac{(2t)^4}{4!}\quad ...) \\
&=1-t^2+\frac{1}{3}t^4\quad ...
\end{split}
\end{equation*}

By substituting \(t=\frac{x}{3}\), we get \begin{equation*}
\begin{split}
\cos^2 \left(\frac{x}{3}\right) &= 1-\left(\frac{x}{3}\right)^2+\frac{1}{3}\left(\frac{x}{3}\right)^4 \\
&=1-\frac{x^2}{9}+\frac{x^4}{243}
\end{split}
\end{equation*} up to and including \(x^4\).

\textbf{b} From part a

\begin{equation*}
\begin{split}
\int_{\frac{\pi}{6}}^{\frac{\pi}{2}} \frac{1}{x} \cos ^2\left(\frac{x}{3}\right) \,dx &= \int_{\frac{\pi}{6}}^{\frac{\pi}{2}} \frac{1}{x} \left(1-\frac{x^2}{9}+\frac{x^4}{243}\right) \,dx \\
&= \int_{\frac{\pi}{6}}^{\frac{\pi}{2}} \frac{1}{x} -\frac{x}{9}+\frac{x^3}{243} \,dx \\
&=\left[ \ln x - \frac{x^2}{18} + \frac{x^4}{972}\right]_{\frac{\pi}{6}}^{\frac{\pi}{2}} \\
&= 0.98295 
\end{split}
\end{equation*}

\end{tcolorbox}




\end{document}
